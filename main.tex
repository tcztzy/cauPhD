% \documentclass[10pt]{beamer}
\documentclass[xcolor=svgnames, t, aspectratio=169]{ctexbeamer}
\usepackage{booktabs}
\usepackage{makecell}

\usetheme[
%%% options passed to the outer theme
%    progressstyle=corner,   % either fixedCircCnt, movCircCnt, or corner
%    rotationcw,                 % change the rotation direction from counter-clockwise to clockwise
%    shownavsym                  % show the navigation symbols
  ]{tarusimple}

\input{settings/packages.tex}
%% 自定义相关的名称宏命令
%% ==================================================
%% \newcommand{\yourcommand}[参数个数]{内容}
% 塔里木大学各单位名称
\newcommand{\taru}{塔里木大学}
\newcommand{\cie}{信息工程学院}
\newcommand{\ciee}{信息与电气工程学院}

%% 签署春秋学期日期命令
\newcommand{\tomonth}{
  \the\year 年\the\month 月
}


\newcommand{\tomonthen}{
  \ifcase\the\month
  \or January%
  \or February%
  \or March%
  \or April%
  \or May%
  \or June%
  \or July%
  \or August%
  \or September%
  \or October%
  \or November%
  \or December%
  \fi, \the\year
}

\newcommand{\tosemester}{
  \the\year 年\ 
  \ifcase\the\month
  \or 秋%
  \or 春%
  \or 春%
  \or 春%
  \or 春%
  \or 春%
  \or 春%
  \or 夏%
  \or 秋%
  \or 秋%
  \or 秋%
  \or 秋%
  \fi
}

\newcommand{\tosemesteren}{
  \ifcase\the\month
  \or Autumn%
  \or Spring%
  \or Spring%
  \or Spring%
  \or Spring%
  \or Spring%
  \or Summer%
  \or Autumn%
  \or Autumn%
  \or Autumn%
  \or Autumn%
  \or Autumn%
  \fi, \the\year
}

% 插图路径设置
% ==================================================
\graphicspath{{figs/}}%图片所在的目录
% ==================================================

% 为标题页/封底页指定一个 logo
\pgfdeclareimage[height=0.5cm]{titlepagelogo}{cauname}% 标题页
\titlegraphic{% 标题页底部
  \pgfuseimage{titlepagelogo}
}


% 载入需要的TiKZ库
\usetikzlibrary{chains}

%% 设置绘制目录结构的宏及参数
\usepackage[edges]{forest}
\definecolor{folderbg}{RGB}{124,166,198}
\definecolor{folderborder}{RGB}{110,144,169}
\newlength\Size
\setlength\Size{4pt}
\tikzset{%
  folder/.pic={%
      \filldraw [draw=folderborder, top color=folderbg!50, bottom color=folderbg] (-1.05*\Size,0.2\Size+5pt) rectangle ++(.75*\Size,-0.2\Size-5pt);
      \filldraw [draw=folderborder, top color=folderbg!50, bottom color=folderbg] (-1.15*\Size,-\Size) rectangle (1.15*\Size,\Size);
    },
  file/.pic={%
      \filldraw [draw=folderborder, top color=folderbg!5, bottom color=folderbg!10] (-\Size,.4*\Size+5pt) coordinate (a) |- (\Size,-1.2*\Size) coordinate (b) -- ++(0,1.6*\Size) coordinate (c) -- ++(-5pt,5pt) coordinate (d) -- cycle (d) |- (c) ;
    },
}
\forestset{%
declare autowrapped toks={pic me}{},
declare boolean register={pic root},
pic root=0,
pic dir tree/.style={%
for tree={%
    folder,
    %font=\ttfamily,
    grow'=0,
    s sep=1.0pt,
    font=\small \sffamily,
    %fit=band,
    %ysep = 1.0pt,
    inner ysep = 2.6pt,
  },
before typesetting nodes={%
for tree={%
edge label+/.option={pic me},
},
if pic root={
tikz+={
\pic at ([xshift=\Size].west) {folder};
},
align={l}
}{},
},
},
pic me set/.code n args=2{%
    \forestset{%
      #1/.style={%
          inner xsep=2\Size,
          pic me={pic {#2}},
        }
    }
  },
pic me set={directory}{folder},
pic me set={file}{file},  
}
%% ==================================================


%%% Local Variables: 
%%% mode: latex
%%% TeX-master: "../main.tex"
%%% End: 
  
  
\title{中国农业大学博士入学复试报告}
%\subtitle{基于冠层改进的Cotton2K模型}
\date[2022/4/24]{\zhdate{2022/4/24}}
\author[唐梓涯]
{唐梓涯}
\institute
{
  \ciee
  
}

\begin{document}

{\taruwavesbg%
\begin{frame}[plain,noframenumbering]
  \titlepage
\end{frame}
}

\begin{frame}{目录}{内容列表}
  \zihao{2}
  \begin{enumerate}
    \item 个人基本情况
    \item 教育经历
    \item 工作经历
    \item 科研经历及取得成果
    \item 其他奖励、荣誉
  \end{enumerate}
\end{frame}

\begin{frame}{个人基本情况}{简介}
  \zihao{-1}唐梓涯,男,汉族,1992 年 3 月出生,江苏如东人,中共预备党员。
\end{frame}

\begin{frame}{教育经历}{毕业院校、专业}
  \begin{table}
    %\caption{本科、硕士就读院校及专业}
    \begin{tabular}{cc}
      本科                                       & 硕士                                                           \\
      \includegraphics[scale=1.15]{ppcnwafu.png} & \includegraphics[scale=0.5,trim=88 71 66 77,clip]{cietaru.png} \\
      2010{-}2014                                & 2019至今\\
      植物保护 (090103)                          & 农业电气化与自动化 (082804)
    \end{tabular}
  \end{table}
\end{frame}
\begin{frame}{工作经历}{南京振古}
  \begin{center}
    \includegraphics[scale=0.18]{zettage.png}
  \end{center}
\end{frame}
\begin{frame}{工作经历}{Microsoft}
  \includegraphics[scale=0.14,trim=0 0 0 120,clip]{microsoft_css.jpg}
  \includegraphics[scale=0.2]{azure.png} 
\end{frame}
\begin{frame}{工作经历}{科大讯飞}
  \begin{center}
    \includegraphics[scale=0.2,trim=0 0 0 300,clip]{iflytek.jpg}
  \end{center}
\end{frame}
\begin{frame}{科研经历}{维语语料库}
  \begin{center}
    \includegraphics[scale=0.2]{db65t3690-2015.png}
    \includegraphics[scale=0.16]{uyghurpypi.png}
  \end{center}
\end{frame}
\begin{frame}{科研经历}{无膜棉生长模拟}
  \begin{center}
    \includegraphics[scale=0.16, trim=0 100 0 0, clip]{yield.jpg}
  \end{center}
\end{frame}
\begin{frame}{科研经历}{无膜棉生长模拟}
  \begin{center}
    \includegraphics[scale=0.11, angle=270, origin=c]{field.jpg}
    \includegraphics[scale=0.16]{cotton2kpypi.png}
  \end{center}
\end{frame}
\begin{frame}{科研经历}{成果}
  一篇题为 \textit{Modeling non-mulched cultivation cotton growth and yield responses to irrigation scheduling using canopy-modified Cotton2K model} 论文投稿期刊\textit{Computers and Electronics in Agriculture},当前状态为 Under review.
  \begin{center}
    \includegraphics[scale=1.5]{compag.jpg}
    \includegraphics[scale=0.18,trim=1 0 0 1, clip]{submission.png}
  \end{center}
\end{frame}
\begin{frame}{其他奖励、荣誉}{列表}
  \zihao{3}
  \begin{itemize}
    \item 2019 年“华为杯”第十六届研究生数学建模大赛三等奖
    \item 2021 年“华为杯”第十八届研究生数学建模大赛三等奖
    \item 2020 年塔里木大学优秀研究生干部
    \item 2021 年“蓝桥杯”个人软件赛 Python 大学生组全国总决赛三等奖
  \end{itemize}
\end{frame}
\begin{frame}{其他奖励、荣誉}{开源}
  \begin{center}
    \raisebox{-0.5\height}{\includegraphics[scale=0.2]{ghprofile.png}}
    \raisebox{-0.5\height}{\includegraphics[scale=0.5]{gh.png}}
  \end{center}
\end{frame}
{\taruwavesbg
\begin{frame}[plain,noframenumbering]
  \finalpage{谢谢各位老师聆听!}
\end{frame}
}

\end{document}
